\documentclass{article}

\usepackage{multirow}
\usepackage{graphicx}



\title{Design and Simulation of a USB 2.0 Transceiver in SystemVerilog}
\author{Andrew Lu, Jiayi Liu, Leo Lesmes, Nhan Do}

\graphicspath{{images/}}


\begin{document}
\maketitle

\begin{abstract}
With SystemVerilog code, model a USB 2.0 Transceiver Macrocell Interface (UTMI) that acts as  takes in data packets from a Serial Interface Engine (SIE) and outputs.With SystemVerilog code, model a USB 2.0 Transceiver Macrocell Interface (UTMI) that acts as  takes in data packets from a Serial Interface Engine (SIE) and outputs
.With SystemVerilog code, model a USB 2.0 Transceiver Macrocell Interface (UTMI) that acts as  takes in data packets from a Serial Interface Engine (SIE) and outputs
With SystemVerilog code, model a USB 2.0 Transceiver Macrocell Interface (UTMI) that acts as  takes in data packets from a Serial Interface Engine (SIE) and outputs
With SystemVerilog code, model a USB 2.0 Transceiver Macrocell Interface (UTMI) that acts as  takes in data packets from a Serial Interface Engine (SIE) and outputs

\end{abstract}

\section{Technical}
\subsection{Overview}
We are designing the part in green, the 
\begin{figure}[htp]
    \centering
    \includegraphics[width=8cm]{asic_functional_blocks}
    \caption{ASIC Functional Blocks}
    \label{fig:hl}
\end{figure}
Low speed 

\begin{figure}[htp]
    \centering
    \includegraphics[width=12cm]{RTL}
    \caption{UTMI RTL Diagram}
    \label{fig:rtl}
\end{figure}

\subsubsection*{Hardware \& Software}
Numerous pieces of software were used to help develop this project, along with mutliple coding languages. The names of the products used, along with their purpose in the project, are listed below.
\begin{table}[htp]
    %\centering
    \begin{tabular}{ |l|l|l| } 
        \hline
        \multicolumn{1}{|c|}{\textbf{Type}} & \multicolumn{1}{c|}{\textbf{Name}} & \multicolumn{1}{c|}{\textbf{Purpose}} \\
        \hline
        \multirow{6}{6em}{Software} & Google Docs & Research/brainstorming \\
        & Figma & Research/brainstorming, and creating diagrams \\ 
 
        & Google Slides & Slideshow software that the Final Presentation was created in. \\ 
        & Visual Studio Code & IDE used for coding the project and writing the final report. \\
        & Github & Project was hosted and shared \\ 
        & MyFPGA & In web-browser simulation tool for testing SystemVerilog code \\ 
        \hline
        \multirow{2}{6em}{Code Language} & SystemVerilog & Language that the UTMI was written in \\ 
        & LaTeX & Final project report (this document) was written in LaTeX \\ 
        \hline
\end{tabular}

\caption{Products used to develop the project}
\label{table:1}
\end{table}

\subsubsection*{Division of Labor}
\paragraph{Everyone:} Researched the various aspects of USB 2.0, drew diagrams, debugged and final testing, wrote final presentation, designed final presentation. Presented final presentation to class.
\paragraph{Andrew:} Coded the
\paragraph{Jiayi:} Coded th
\paragraph{Leo:} Coded the . Main person for Final Report and Presentaion.
\paragraph{Nhan:} Coded the . Worked to integrate 

\subsubsection*{Timeline}
Large picture of the timeline to be placed here
\subparagraph{Week 0 : Project Proposal Drafting}
\subparagraph{Week 1-2 : Research}
\subparagraph{Week 3-4 : Diagramming and Planning}
\subparagraph{Week 5-7: Coding and Testing}
\subparagraph{Week 9 : Presentation and Paper Creation}




\subsection{Design}
\paragraph{RX State Machine}
\paragraph{TX State Machine}
\paragraph{DPLL}
\paragraph{EOP Detector}
\paragraph{NZRI Decoder/Encoder}
\paragraph{Bit Stuffing/Unstuffing}
\paragraph{SIPO and PISO Registers}

\subsection{Simulation}
\subsubsection*{Testing \& Benchmarks}
\subsubsection*{Operation Documentation}
\subsubsection*{Expansion/Next Steps}


\section{Reflection}
\subsection*{Learning Outcomes}
Helped with SystemVerilog knowledge
Helped with reading technical documentation
\subsection*{Preferences}
\paragraph{Andrew:}
\paragraph{Jiayi:}
\paragraph{Leo:}
\paragraph{Nhan:}
\subsection*{Challenges}
Learning complicated concepts.
\newline
Doing a larger project in a newly learned language.
\newline
Managing the timeline over a long period with other to-dos
\newline
Communicating/organizing progress with each other.
\end{document}